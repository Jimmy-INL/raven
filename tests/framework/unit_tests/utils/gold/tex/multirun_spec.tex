\subsection{MultiRun}
   The \textbf{MultiRun} step allows the user to assemble the calculation flow of an analysis that
  requires multiple ``runs'' of the same model. This step is used, for example, when the input
  (space) of the model needs to be perturbed by a particular sampling strategy. The specifications
  of this type of step must be defined within a \xmlNode{MultiRun} XML block.

  The \xmlNode{MultiRun} node recognizes the following parameters:
    \begin{itemize}
      \item \xmlAttr{verbosity}: \xmlDesc{[silent, quiet, all, debug], optional}, 
        Desired verbosity of messages coming from this entity
      \item \xmlAttr{name}: \xmlDesc{string, required}, 
        User-defined name to designate this entity in the RAVEN input file.
      \item \xmlAttr{sleepTime}: \xmlDesc{float, optional}, 
        Determines the wait time between successive iterations within this step, in seconds.
      \item \xmlAttr{re-seeding}: \xmlDesc{string, optional}, 
         this optional attribute could be used to control the seeding of the random number generator
        (RNG). If inputted, the RNG can be reseeded. The value of this attribute can be: either 1)
        an integer value with the seed to be used (e.g. \xmlAttr{re-seeding} = ``20021986''), or 2)
        string value named ``continue'' where the RNG is not re-initialized
      \item \xmlAttr{pauseAtEnd}: \xmlDesc{string, optional}, 
        -- no description yet --
      \item \xmlAttr{fromDirectory}: \xmlDesc{string, optional}, 
        -- no description yet --
      \item \xmlAttr{repeatFailureRuns}: \xmlDesc{string, optional}, 
        -- no description yet --
  \end{itemize}

  The \xmlNode{MultiRun} node recognizes the following subnodes:
  \begin{itemize}
    \item \xmlNode{Input}: \xmlDesc{string}, 
      Inputs to the step operation
      The \xmlNode{Input} node recognizes the following parameters:
        \begin{itemize}
          \item \xmlAttr{class}: \xmlDesc{string, required}, 
            -- no description yet --
          \item \xmlAttr{type}: \xmlDesc{string, required}, 
            -- no description yet --
      \end{itemize}

    \item \xmlNode{Model}: \xmlDesc{string}, 
      Entity containing the model to be executed
      The \xmlNode{Model} node recognizes the following parameters:
        \begin{itemize}
          \item \xmlAttr{class}: \xmlDesc{string, required}, 
            -- no description yet --
          \item \xmlAttr{type}: \xmlDesc{string, required}, 
            -- no description yet --
      \end{itemize}

    \item \xmlNode{Sampler}: \xmlDesc{string}, 
      Entity containing the sampling strategy
      The \xmlNode{Sampler} node recognizes the following parameters:
        \begin{itemize}
          \item \xmlAttr{class}: \xmlDesc{string, required}, 
            -- no description yet --
          \item \xmlAttr{type}: \xmlDesc{string, required}, 
            -- no description yet --
      \end{itemize}

    \item \xmlNode{Output}: \xmlDesc{string}, 
      Entity to store results of the step
      The \xmlNode{Output} node recognizes the following parameters:
        \begin{itemize}
          \item \xmlAttr{class}: \xmlDesc{string, required}, 
            -- no description yet --
          \item \xmlAttr{type}: \xmlDesc{string, required}, 
            -- no description yet --
      \end{itemize}

    \item \xmlNode{Optimizer}: \xmlDesc{string}, 
      Entity containing the optimization strategy
      The \xmlNode{Optimizer} node recognizes the following parameters:
        \begin{itemize}
          \item \xmlAttr{class}: \xmlDesc{string, required}, 
            -- no description yet --
          \item \xmlAttr{type}: \xmlDesc{string, required}, 
            -- no description yet --
      \end{itemize}

    \item \xmlNode{SolutionExport}: \xmlDesc{string}, 
      Entity containing auxiliary output for the solution of this step
      The \xmlNode{SolutionExport} node recognizes the following parameters:
        \begin{itemize}
          \item \xmlAttr{class}: \xmlDesc{string, required}, 
            -- no description yet --
          \item \xmlAttr{type}: \xmlDesc{string, required}, 
            -- no description yet --
      \end{itemize}

    \item \xmlNode{Function}: \xmlDesc{string}, 
      Functional definition for use within this step
      The \xmlNode{Function} node recognizes the following parameters:
        \begin{itemize}
          \item \xmlAttr{class}: \xmlDesc{string, required}, 
            -- no description yet --
          \item \xmlAttr{type}: \xmlDesc{string, required}, 
            -- no description yet --
      \end{itemize}
  \end{itemize}
